\documentclass[letterpaper, onecolumn, 11pt]{report}
\usepackage[utf8]{inputenc}
\usepackage[spanish]{babel}
\usepackage{amsmath}
\usepackage{graphicx}
\usepackage{physics}
\usepackage{wrapfig}
\usepackage{hyperref}
\usepackage{marginnote}
\usepackage{caption}
\usepackage{amsfonts}
%\usepackage{draculatheme} %Agregar draculatheme.sty  al directorio del proyecto LaTeX
\spanishdecimal{.}
\renewcommand{\theenumi}{\alph{enumi}}
\usepackage{hyperref}
\usepackage{marginnote}
\hypersetup{colorlinks=true, linkcolor=red}

%\renewcommand*{\marginnotevadjust}{-0.1cm}
%\renewcommand*{\marginfont}{\footnotesize}
%\usepackage[right=4.5cm,left=2cm,top=3cm,bottom=3.0cm]{geometry}
\begin{document}
\section*{1}
Calcular las siguientes integrales múltiples, sobre el rectángulo $R$ indicado en cada caso.
\begin{enumerate}
	\item $\int\int_R6xy^2\dd A, \qquad R=[2,4]\times [1,2]$
Esta integral se puede escribir como
\begin{equation}
	\int_1^2\int_2^46xy^2\dd x\dd y=84
\end{equation}
\item $\int\int_R(2x-4y^3)\dd A, \qquad R=[-5,4]\times[0,3]$
\begin{equation}
	\int_0^3\int_{-5}^4(2x-4y^3)\dd x\dd y=-756
\end{equation}
\item $\int\int_R\frac{1}{(2x+3y)^2}\dd A\quad R=[0,1]\times[1,2]$
\begin{equation}
	\int_0^1\int_1^2\frac{1}{(2x+3y)^2}\dd x\dd y
\end{equation}
\item $\int\int\int_R8xyz\dd V \quad R=[2,3]\times [1,2]\times [0,1] $
\begin{equation}
	\int_0^1\int_1^2\int_2^3 8xyz\dd x\dd y\dd z = 15
\end{equation}
	
\end{enumerate}
\section*{2}
Calcular la integral $\int\int_D\frac{y}{(x^2+y^2)^{5/2}}\dd A$ donde $D$ es la región arriba del eje $x$, acotada  a la izquerda por la recta $x=1$ y arriba por la curva $x^2+y^2=2$
\begin{equation}
	\int_1^2\int_0^{\pi/4}\frac{\sin\theta}{r^3}\dd \theta \dd r=\frac{3(\sqrt{2}-1)}{8\sqrt{2}}
\end{equation}


\section*{3}
Calcular la integral $\int\int_D\frac{y^2}{(x^2+y^2)^{1/2}}\dd A$ donde $D$ es la región del círculo $x^2+y^2=2x$, arriba del eje $x$ y a la derecha de la recta $x=1$
\begin{equation}
	x^2+y^2=2x = (x-1)^2+y^2=1
\end{equation}
\begin{equation}
	\int_1^2\int_0^{2\pi}\frac{r^2\sin^2\theta}{r}r\dd\theta \dd r=\frac{\pi}{4}
\end{equation}







\end{document}

