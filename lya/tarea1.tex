\documentclass[letterpaper, onecolumn, 12pt]{report}
\usepackage[utf8]{inputenc}
\usepackage[spanish]{babel}
\usepackage{amsmath}
\usepackage{graphicx}
%\usepackage{wrapfig}
\usepackage{hyperref}
%\usepackage{marginnote}
\usepackage{physics}
%\usepackage{cancel}
%\usepackage{subfig}
\usepackage{caption}
\spanishdecimal{.}

%\renewcommand*{\marginnotevadjust}{-0.1cm}
%\renewcommand*{\marginfont}{\footnotesize}
%\usepackage[right=4.5cm,left=2cm,top=3cm,bottom=3.0cm]{geometry}
\renewcommand{\theenumi}{\alph{enumi}}
\usepackage[right=3.0cm,left=1.20cm,top=1.5cm,bottom=1.5cm]{geometry}

\hypersetup{colorlinks=true, linkcolor=red}
\begin{document}
\sffamily

\title{\vspace{-10mm} Tarea Lya}

\author{Borja Diez\\
         }% <-this % stops a space

\makeatletter
\def\@maketitle{
\begin{center}
{\Huge \bfseries \sffamily \@title }\\[4ex]
Entregado el: \@date\\
{\normalsize \@author}\\[4ex]
%\@date\\[8ex]
\end{center}}
\makeatother

%maketitle
\section*{Problema 1}


Calculemos primero la derivada parcial de $r$ con respecto a $x$ e $y$:
\begin{align*}
\frac{\partial r}{\partial x}=\frac{\partial}{\partial x}[(x^2+y^2)^{1/2}]=\frac{2x}{2}(x^2+y^2)^{-1/2}=\frac{2x}{2\sqrt{x^2+y^2}}=\frac{x}{\sqrt{x^2+y^2}}
\end{align*}

\begin{align*}
    \frac{\partial r}{\partial y}=\frac{\partial}{\partial y}[(x^2+y^2)^{1/2}]=\frac{2y}{2}(x^2+y^2)^{-1/2}=\frac{2y}{2\sqrt{x^2+y^2}}=\frac{y}{\sqrt{x^2+y^2}}
\end{align*}
    
Luego
\begin{align*}
    \left(\pdv{r}{x}\right)^2+\left(\pdv{r}{y}\right)^2&=\left(\frac{x}{\sqrt{x^2+y^2}}\right)^2+\left(\frac{y}{\sqrt{x^2+y^2}}\right)^2\\
    &=\frac{x^2}{x^2+y^2}+\frac{y^2}{x^2+y^2}\\
    &=\frac{x^2+y^2}{x^2+y^2}\\
    &=1
\end{align*}
Así
\begin{equation}\label{con1}
    \boxed{\left(\pdv{r}{x}\right)^2+\left(\pdv{r}{y}\right)^2=1}
\end{equation}

Calculemos ahora las segundas derivadar parciales de $r$ con respcto a $x$ e $y$
\begin{align*}
    \pdv[2]{r}{x}&=\frac{\partial}{\partial x}\left[\frac{x}{\sqrt{x^2+y^2}}\right]\\
    &=\frac{\sqrt{x^2+y^2}-x\frac{x}{\sqrt{x^2+y^2}}}{x^2+y^2}\\
    &=\frac{\sqrt{x^2+y^2}-\frac{x^2}{\sqrt{x^2+y^2}}}{x^2+y^2}\\
    &=\frac{\frac{x^2+y^2-x^2}{\sqrt{x^2+y^2}}}{x^2+y^2}\\
    &=\frac{y^2}{(x^2+y^2)\sqrt{x^2+y^2}}
\end{align*}

De manera similar tenemos
\begin{align*}
    \pdv[2]{r}{y}=\frac{\partial}{\partial y}\left[\frac{y}{\sqrt{x^2+y^2}}\right]=\frac{x^2}{(x^2+y^2)\sqrt{x^2+y^2}}
\end{align*}
Luego,
\begin{align*}
    \pdv[2]{r}{x}+\pdv[2]{r}{y}&=\frac{y^2}{(x^2+y^2)\sqrt{x^2+y^2}}+\frac{x^2}{(x^2+y^2\sqrt{x^2+y^2})}\\
    &=\frac{x^2+y^2}{(x^2+y^2)\sqrt{x^2+y^2}}\\
    &=\frac{1}{\sqrt{x^2+y^2}}\\
    &=\frac{1}{r}
\end{align*}
Es decir,
\begin{equation}\label{con2}
    \boxed{\pdv[2]{r}{x}+\pdv[2]{r}{y}=\frac{1}{r}}
\end{equation}

Calculemos ahora las segundas derivadas parciales de $u$ con respecto a $x$ e $y$
\begin{align*}
    \pdv{u}{x}=\pdv{u}{r}\pdv{r}{x}=f'(r)\pdv{r}{x}
\end{align*}
Luego 
\begin{align}
    \nonumber \pdv[2]{u}{x}=\frac{\partial}{\partial x}\left[f'(r)\pdv{r}{x}\right]&=\pdv{f'(r)}{x}\pdv{r}{x}+f'(r)\pdv[2]{r}{x}\\ 
   \nonumber &=\pdv{r}{x}\pdv{f'(r)}{r}\pdv{r}{x}+f'(r)\pdv[2]{r}{x}\\ 
    &=f''(r)\left(\pdv{r}{x}\right)^2+f'(r)\pdv[2]{r}{x} \label{1}
\end{align}
De manera similar tenemos
\begin{equation}\label{2}
    \pdv[2]{u}{y}=f''(r)\left(\pdv{r}{y}\right)^2+f'(r)\pdv[2]{r}{y}
\end{equation}
Sumando (\ref{1}) y (\ref{2}) obtenemos
\begin{align*}
    \pdv[2]{u}{x}+\pdv[2]{u}{y}&=f''(r)\left(\pdv{r}{x}\right)^2+f'(r)\pdv[2]{r}{x}+f''(r)\left(\pdv{r}{y}\right)^2+f'(r)\pdv[2]{r}{y}\\
    &=f''(r)\left[\left(\pdv{r}{x}\right)^2+\left(\pdv{r}{y}\right)^2\right] + f'(r)\left[\pdv[2]{r}{x}+\pdv[2]{r}{y}\right]
\end{align*}
Utilizando (\ref{con1}) y (\ref{con2}) tenemos
\begin{equation*}
   \boxed{\pdv[2]{u}{x}+\pdv[2]{u}{y}=f''(r)+\frac{1}{r}f'(r)}
\end{equation*}

\end{document}
