\begin{problema}
Dos cilindros largos (radio $a$ y $b$) estan separados por un materia con conductividad $\sigma$. Si estan mantenidos a una diferencia de potencial $V$, ¿qúe corriente fluye de uno hacia el otro, en una longitud $L$?
\end{problema}
\begin{figure}[h!]
	\begin{center}
		\includegraphics[scale=0.5]{fig/02.png}
	\end{center}	
\end{figure}

\begin{sol}
\end{sol}

Por ley de Gauss, el campo generado por el cilinro en para $r>a$ es
\begin{equation}
	\vb*{E}=\frac{\lambda}{2\pi \epsilon_0 r}\vu*{r}
\end{equation}
Notar que es tambien es el campo entre ambos cilindros.

La corriente viene dada por
\begin{align}
	I=\int\vb*{J}\cdot \dd\vb*{a}&=\sigma\int\vb*{E}\cdot \dd \vb*{a}\\
														 &=\frac{\sigma}{\epsilon_0}\lambda L\\
														 &=\frac{\lambda}{2\pi \epsilon_0}\int_{-L/2}^{L/2}\int_0^{2\pi}\frac{s}{s}\dd\phi\dd z\\
														 &=\frac{\sigma \lambda L}{\epsilon_0}
\end{align}

La diferencia de potencual entre los cilindros es
\begin{align}
	V&=-\int_b^{a}\vb*{E}\cdot \dd \vb*{l}\\
	 &=\frac{\lambda}{2\pi\epsilon_0}\int_{a}^b\frac{\dd s}{s}\\
	 &=\frac{\lambda}{2\pi\epsilon_0}\ln\left(\frac{b}{a}\right)
\end{align}
Despejando $\lambda$
\begin{equation}
	\lambda=\frac{2\pi\epsilon_0}{ln(b/a)}V
\end{equation}

Reemplazando en la expresión para la corriente obtenemos
\begin{equation}
	\boxed{I=\frac{2\pi \sigma L}{ln(b/a)}V}
\end{equation}

